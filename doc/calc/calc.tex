% !TEX TS-program = arara
% arara: xelatex: { synctex: on, options: [-halt-on-error] } 
% arara: biber
% arara: biber: { options: [--noremove-tmp-dir] }
% % arara: texindy: { markup: xelatex }
% %% arara: makeglossaries
% arara: xelatex: { synctex: on, options: [-interaction=batchmode, -halt-on-error] }
% arara: xelatex: { synctex: on, options: [-interaction=batchmode, -halt-on-error]  }
% % arara: clean: { extensions: [ aux, log, out, run.xml, ptc, toc, mw, synctex.gz, ] }
% % arara: clean: { extensions: [ bbl, bcf, blg, ] }
% % arara: clean: { extensions: [ glg, glo, gls, ] }
% % arara: clean: { extensions: [ idx, ilg, ind, xdy, ] }
% % arara: clean: { extensions: [ plCode, plData, plMath, plExercise, plNote, plQuote, ] }
%-----------------------------------------------------------------
\documentclass[12pt]{PalisadesLakesBook}
% \geomHDTV
% \geomLandscape
\geomHalfDTV
\geomPortraitOneColumn
%-----------------------------------------------------------------

%\AsanaFonts % misssing \mathhyphen; less on page than Cormorant/Garamond
%\CormorantFonts % light, missing unicode greek
\EBGaramondFonts % fewest pages
%\ErewhonFonts
%\FiraFonts % tall lines, all sans, much less per page, missing \in?
%\GFSNeohellenicFonts 
%\KpFonts
%\LatinModernFonts
%\LegibleFonts
%\LibertinusFonts
%\NewComputerModernFonts
%\STIXTwoFonts
%\BonumFonts % most pages
%\PagellaFonts
%\ScholaFonts
%\TermesFonts
%\XITSFonts

%-----------------------------------------------------------------
\togglefalse{plMath}
\togglefalse{plCode}
\togglefalse{plData}
\togglefalse{plNote}
\togglefalse{plExercise}
\togglefalse{plQuote}
\togglefalse{printglossary}
\togglefalse{printindex}
%-----------------------------------------------------------------
\title{Computable mathematics:
real numbers, calculus, analysis, etc.}
\author{John Alan McDonald 
(palisades dot lakes at gmail dot com)}
\date{draft of \today}
%-----------------------------------------------------------------
\begin{document}

\maketitle
\PalisadesLakesTableOfContents{7}
%-----------------------------------------------------------------
\begin{plSection}{Introduction} 
 

\end{plSection}%{Introduction}
%-----------------------------------------------------------------
\begin{plSection}{Generic functions} 

In this section, I intend to present generic function 
from something like first principles.

Generic functions are often presented as some kind of unusual 
addons to ``normal'' object-oriented languages.
An example of this is the choice of  naming in Clojure,
which implies that there are ``normal'' (single-dispatch)
methods, and then there are these odd things called 
``multimethods'', 
which are poorly integrated into the language,
and rarely used, as far as I can tell.
This may be due to the message-passing metaphor,
common in early object-oriented languages,
but whatever the reason, 
I think it's worth coming at this from a different direction:

(Languages where all functions are generic: 
Dylan~\cite{ShalitPiazzaMoon:1992:Dylan}, 
Julia~\cite{Bezanson:2017:Julia}, \ldots)

It is common for procedures to begin by analyzing their inputs,
and using the result to choose a specific algorithm.

\begin{plExample}{{\javaFont BigInteger.multiply(BigInteger)}
}{BigInteger.multiply}
An example of this is 
{\javaFont 
BigInteger.multiply(BigInteger)}~\cite{BigInteger.multiply:2021}.

Instances of {\javaFont BigInteger} can represent 
integers much larger than can be held in the fixed length
representations 
({\javaFont byte}/{\javaFont Byte}\ldots
{\javaFont long}/{\javaFont Long}).
The Java $17$ API~\cite{Oracle:2021:JDK17} 
guarantees at least numbers in the range
$\left[ -2^{{2^{31}}-1}, 2^{{2^{31}}-1} \right]$,
and the reference implementation,
 {\javaFont ArithmeticException}s for operations whose result
would exceed it.
The implementation is essentially a bit sequence 
for the magnitude, 
plus a flag for positive or negative.

{\javaFont BigInteger.multiply(BigInteger)} 
is a procedure which computes the product of $2$ 
{\javaFont BigInteger}s.
It chooses among $8$--$12$ `methods' (depending on how you want to
count):
\begin{enumerate}
  \item It first checks if either input number if zero,
returning an existing {\javaFont ZERO} singleton in that case.

\item It then checks whether both inputs are the same 
{\javaFont BigInteger} instance (and have more 
than a certain number of significant bits),
in which case it hands off to a similar  procedure optimized
for \textit{squaring}---multiplying  a number by itself.

\item It then chooses sequentially from $4$ `methods', 
depending on the number of
significant bits in \textit{both} arguments:
\begin{enumerate}
  \item Either argument fits in a $32$ bit {\javaFont int}.
  \item Either argument is below an upper bound for
  \textit{grade-school} 
  multiplication~\cite{wiki:GradeSchoolMultiplication}.
  \item Both arguments are above the grade-school upper bound 
  and below an upper bound for \textit{Karatsuba}
   multiplication~\cite{wiki:KaratsubaMultiplication}.
  \item Otherwise (both arguments above the grade-school cap
  and at least one above the Karatsuba cap)
  use \textit{Toom-Cook} 
  multiplication~\cite{wiki:ToomCookMultiplication}.
\end{enumerate}
\end{enumerate}

Note that all these tests depend on \textit{both} arguments,
and depend on their state/value, rather than type/class.
\end{plExample}%{{\javaFont BigInteger.multiply}}{BigInteger.multiply}}

As of \formatdate{22}{10}{2021},
the most recent change to {\javaFont BigInteger.multiply} 
in the reference implementation was committed more than $3$ years 
earlier (on \formatdate{22}{08}{2018}).
There have only been $6$ commits 
that affected {\javaFont BigInteger.multiply} directly
in the $14$ years
since the reference implementation was initially loaded into 
GitHub on (\formatdate{7}{11}{2007}).

What changes might one want to make to 
{\javaFont BigInteger.multiply}?

\begin{itemize}
\item An obvious one is to add algorithms which are faster for
larger integer values.
The \textit{Sch\"{o}nhage–Strassen} 
algorithm~\cite{wiki:SchonhageStrassenMultiplication}
potentially outperforms Toom-Cook for numbers with more
than $2^{15}$ bits well within the minimum value range
required by the {\javaFont BigInteger} API.
An alternate implementation that adds Sch\"{o}nhage-Strassen
to the sequence of tests choosing between
single {\javaFont int}, grade-school, Karatsuba, and
Toom-Cook can be found at~\cite{tbuktu:2015:EfficientBigIntegerImplementation}.

There are algorithms which are even faster,
for much larger numbers.

\textit{F\"{u}rer}'s algorithm~\cite{ wiki:FurerMultiplication}
potentially outperforms Sch\"{o}nhage-Strassen,
for numbers with more than $2^{64}$ bits, 
outside the range required by the {\javaFont BigInteger} 
API, requiring perhaps $10^{4} \cdots 10^{6}$ 
of the total digital storage---possible, but only if there were
some truly compelling reason.

There are even faster algorithms~\cite{
Hartnett:2019:NLogNIntegerMultiplication,
HarveyVanDerHoeven:2020:NLogNIntegerMultiplication},
but they fall into the class of 
``galactic algorithms''~\cite{wiki:GalacticAlgorithm},
meaning, in this case, they only win as the number of bits
approaches the number of elementary particles in the universe.

\item It would be fairly straightforward to create an unbounded
integer representation, with no significant overhead 
in the range {\javaFont BigInteger}
can handle.
Just use an unbounded sequence of {\javaFont BigInteger}s
for the magnitude, and the grade-school algorithm on the
of {\javaFont BigInteger} words.
This would require a version of {\javaFont BigInteger.multiply}
that returned an overflow amount as well as the truncated
{\javaFont BigInteger} product. 

\item A lot of code would be simplified if it were possible
evaluate general arithmetic expressions without requiring 
specifying which implementation of {\javaFont java.lang.Number}
for the inputs and intermediate values,
while making it possible to specify that if desired.
It would be nice to be able to rely on the system 
to use the most compact/efficient representation for the returned
and intermediate values, especially where it would be possible to
use primitives for inputs and outputs.
\end{itemize}

\end{plSection}%{Generic functions}
%-----------------------------------------------------------------
\begin{plSection}{Arithmetic} 

I suspect that, when most software developers see ``arithmetic'',
they remember elementary school
and think of addition, subtraction, multiplication, and division.
The ``numbers'' passed to these operations are 
understood to be the (classical) reals, 
though probably only partially understood.


\end{plSection}%{Arithmetic}
%-----------------------------------------------------------------
\begin{plSection}{Numbers} 


\end{plSection}%{Numbers}
%-----------------------------------------------------------------
\begin{plSection}{Mathematical structures} 


This adds up to what a mathematician would call an ordered 
\emph{field}~\cite{wiki:FieldMathematics}.
Skipping the ``ordered'' part for now,
a field $\Space{F}$ consists of
$[\Set{F},\Add{\Space{F}},\Mul{\Space{F}},
\Sub{\Space{F}},\Div{\Space{F}},\Zero{\Space{F}}, \One{\Space{F}}]$:
\begin{description}
\item[Elements:] a set $\Set{F}$, the ``numbers''.
\item[Addition:] a binary operation (two argument function)
$\Add{\Space{F}} : \Set{F} \times \Set{F} \rightarrow \Set{F}$
\item[Multiplication:] a binary operation 
$\Mul{\Space{F}} : \Set{F} \times \Set{F} \rightarrow \Set{F}$
\item[Additive inverse:] a unary operation (one argument function)
$\Sub{\Space{F}} :\Set{F} \rightarrow \Set{F}$.
\item[Multiplicative inverse:] 
a unary operation (one argument function)
on everything but the additive identity:
$\Div{\Space{F}} :  \Set{F} \setminus \left\{\Zero{\Space{F}}\right\} 
\rightarrow \Set{F}$.
\item[Additive identity:] a nullary operation 
(zero argument function)
$\Zero{\Space{F}} : \emptyset \rightarrow \Set{F}$.
Mathematicians typically ignore the difference between a zero
argument function and the value of that function,
acting as if $\Zero{\Space{F}}() = \Zero{\Space{F}}\, \in \, \Set{F}$.
\item[Multiplicative identity:] a nullary operation 
$\One{\Space{F}} : \emptyset \rightarrow \Set{F}$.
\end{description}
Note that I am being pedantic about a number of things
that mathematicians usually leave ambiguous,
such as the difference between the field $\Space{F}$ 
and the set of elements $\Set{F}$,
and labeling the operations
with the field for which they are defined.
Another non-standard aspect of my definition is taking
the identities to be zero-argument functions rather than
elements of the set.
The reason for these peculiarities is to better match the needs 
of implementation, which should, I hope, 
become clear later.

The operations satisfy the following laws, for all 
$a, b, c \,\in \, \Set{F}$
\begin{description}

\item[Additive identity:] 
$\Zero{\Space{F}} \,\Add{\Space{F}}\, a \,=\, a$.

\item[Multiplicative identity:] 
$\One{\Space{F}} \,\Mul{\Space{F}}\, a \,=\, a$.

\item[Additive inverse:] 
$\Sub{\Space{F}}(a) \;\Add{\Space{F}}\; a = \Zero{\Space{F}}$.
Binary subtraction is defined as the composition 
of addition and inverse:
$a \,\Sub{\Space{F}}\, b = 
a \,\Add{\Space{F}}\, \left(\Sub{\Space{F}}(b)\right) $

\item[Multiplicative inverse:] 
$\Div{\Space{F}}(a) \;\Mul{\Space{F}}\; a = \One{\Space{F}}$ 
for all $a \neq \Zero{\Space{F}}$.
Binary division is defined as with binary addition,
with the wrinkle that the second argument (the divisor)
cannot be the additive identity.

\item[Associative addition:] 
$\left( a \Add{\Space{F}} b \right)  \Add{\Space{F}} c 
\,=\, 
a  \Add{\Space{F}} \left( b  \Add{\Space{F}} c \right)$

\item[Associative multiplication:] 
$\left( a  \Mul{\Space{F}} b \right) \Mul{\Space{F}} c 
\,=\, 
a \Mul{\Space{F}} \left( b \Mul{\Space{F}} c \right)$

\item[Commutative addition:] 
$a \Add{\Space{F}} b \,=\, b \Add{\Space{F}} a$

\item[Commutative multiplication:] 
$a \Mul{\Space{F}} b \,=\, b \Mul{\Space{F}} a$

\item[Distributive multiplication over addition:] 
$a \, \Mul{\Space{F}} \, \left( b \Add{\Space{F}} c \right)
\,=\, 
\left(a \Mul{\Space{F}} b \right) 
\, \Add{\Space{F}} \, 
\left(a \Mul{\Space{F}} c \right)$

\end{description}

\end{plSection}%{Mathematical structures}
%-----------------------------------------------------------------
\begin{plSection}{Foundations}

The problem with this viewpoint is that computer arithmetic
doesn't work this way
and, more critically, computer arithmetic \emph{can't} work
this way.
The (classical) reals, $\Reals$, are \emph{uncountable};
computer arithmetic can at most handle countably infinite sets
(in the sense that the representable values are 
bounded by available memory (and time),
rather than having an intrinsically limited range)

This suggests the following question:
Is there a countable, implementable number system,
probably a subset of $\Reals$,
which is sufficient for, in my case,
``scientific computing''?

Perhaps surprisingly, the answer to this seems to be yes.
See: 

\citeAuthorYearTitle{Feferman:1989:IsCantorNecessary}

\citeAuthorYearTitle{Feferman:1992:ALittleBit}

Reprinted, updated in:

\citeAuthorYearTitle[ch.~2 ``Is Cantor necessary?'']{
Feferman:1998:LightOfLogic}

\citeAuthorYearTitle[ch.~12 ``Is Cantor necessary? (Conclusion)'']
{Feferman:1998:LightOfLogic}

\citeAuthorYearTitle[ch~14 ``Why a little bit goes a long way:
logical foundations of scientifically applicable mathematics'']{
Feferman:1998:LightOfLogic}

Useful references, intended for undergraduates, include:
 
\citeAuthorYearTitle{Feferman:1989:NumberSystems}

\citeAuthorYearTitle{Henle:2012:RealNumbers}

\citeAuthorYearTitle{Bridger:2019}

In the next few sections I'm going to work thru a series
of number systems, both mathematical and computational.
My approach is to motivate a new structure
by posing a problem that can't be solved in the structures
introduced up to that point.
This leads to an implementation/representation
in terms of previous (structures),
which can then be abstracted into a set of axioms.
The goal is to end up with something that is both
implementable and covers ``scientifically applicable mathematics''.

\begin{plQuote}
{\citeAuthorYearTitle{Thurston:1994:Proof}}{}%
{Mathematics as we practice it is much more formally 
complete and precise than
other sciences, but it is much less formally complete and precise 
for its content than computer programs.
}
\end{plQuote}

The mathematics I will be using is unusual in two ways:

\begin{itemize}
  
\item Universal quantification only.

This is related to, but not exactly the same as,
universal algebra~\cite{wiki:UniversalAlgebra}.
The idea here is that the laws (axioms) 
in the definition of a mathematical structure
will be expressed using
universal (``for all'') quantifiers, 
not existential (``there exists'') ones.

For example, the classical definition of 
\emph{monoid}~\cite{wiki:Monoid} $\Space{M} = [\Set{M}, \odot]$
would be:
\begin{description}
  \item[Elements:] The set $\Set{M}$ is given.
  \item[Binary operation:] The function
  $\odot : \Set{M} \times \Set{M} \rightarrow \Set{M}$ is given.
  Implied here is 
  $m_0 \odot m_1 = \odot(m_0,m_1) \in \Set{M};
  \text{ for all } m_0, m_1 \in \Set{M}$.
  \item[Associativity:] For all $m_0, m_1, m_2 \in \Set{M}$,
  $(m_0 \odot m_1) \odot m_2 \;=\; m_0 \odot (m_1 \odot m_2).$
  \item[Identity:] There exists $e \in \Set{M}$ such that
  $e \odot m \,=\, m \odot e \,=\, m$,
  for all $m \in \Set{M}$. 
\end{description}
The universal version treats the identity as given
(as a nullary/zero-argument/constant) function.
\begin{description}
  \item[Identity:] Given a zero-argument function 
  $e() : \emptyset \rightarrow \Set{M}$ such that
  $e() \odot m \,=\, m \odot e() \,=\, m$,
  for all $m \in \Set{M}$. 
\end{description}
The difference may seem subtle, but the universal approach
helps with test generation.
Proving an implementation satisfies all the laws is difficult 
in either case, but is it easier to validate
universal laws with generative tests.

\item Computable/constructive/intuitionist mathematics

(To be honest, this is not fully worked out.)

The basic idea here is to start from an
idealized, Turing machine equivalent,
virtual machine for ``procedures'', and define 
``computable mathematics'' in terms of what can be computed 
in a finite number of steps.
I believe this restricts us to at most countable sets,
and eliminates some of the paradoxes that led to the crisis
in the foundations of mathematics circa 1900.
More important in a practical sense, a mathematics that is
defined by finitary procedures translates more-or-less
directly into physical computation---the only difference is 
that real machines are bounded in space and time.

There is at least some reason to think 
that this kind of foundation covers all ``scientifically
applicable mathematics''.
See:\\
\citeAuthorYearTitle{Feferman:1989:IsCantorNecessary}\\
\citeAuthorYearTitle{Feferman:1992:ALittleBit}\\
(Reprinted, updated in:\\
\citeAuthorYearTitle[ch.~2 ``Is Cantor necessary?'']{
  Feferman:1998:LightOfLogic}\\
\citeAuthorYearTitle[ch.~12 ``Is Cantor necessary? (Conclusion)'']
{Feferman:1998:LightOfLogic}\\
\citeAuthorYearTitle[ch~14 ``Why a little bit goes a long way:
logical foundations of scientifically applicable mathematics'']{
  Feferman:1998:LightOfLogic})\\
\citeAuthorYearTitle{Aberth:2001:ComputableCalculus}\\

This approach to mathematics eliminates much of the ugliness
of classical analysis---no axiom of choice, no unmeasurable
sets, etc.

However, more or less in exchange, it adds other complexities,
related to the fact that one can't in general tell whether a given
procedure with a given input, will ever halt.
This modifies our idea of ``true'' and ``false''.
A statement is ``true'' if there is a finite procedure that
proves it true, and the same for false,
but we have a third state: not halting. 
And we can't be sure whether a running procedure won't halt,
or is just very slow.

\end{itemize}

\begin{plSection}{\RCA{0}}

(This section will most likely not appear in a final version.)

  `The initials ``${RCA}$'' stand for ``recursive comprehension axiom'',
  where ``recursive'' means ``computable'',
  as in recursive function.
  This name is used because \RCA{0}
  corresponds informally to ``computable mathematics''.'~\cite{wiki:ReverseMathematics}

  \begin{plQuote}
  {\citeAuthorTitle{wiki:ReverseMathematics}}%
  {foo}%
  {%
    Despite its seeming weakness (of not proving any non-computable sets exist), 
    \RCA{0} is sufficient to prove a number of classical theorems which, therefore, 
    require only minimal logical strength. 
    These theorems are, in a sense, below the reach of the reverse mathematics enterprise 
    because they are already provable in the base system. 
    The classical theorems provable in \RCA{0} include:
  \begin{itemize}
    \item Basic properties of the natural numbers, integers, and rational numbers (for example,
    that the latter form an ordered field).
    \item Basic properties of the real numbers (the real numbers are an Archimedean ordered field;
    any nested sequence of closed intervals whose lengths tend to zero has a single point in its intersection;
    the real numbers are not countable)~\cite[Section II.4]{Simpson:2009:Subsystems}.
    \item The Baire category theorem for a complete separable metric space
    (the separability condition is necessary to even state the theorem in the language of second-order
    arithmetic)~\cite[Theorem II.5.8]{Simpson:2009:Subsystems}.
    \item The intermediate value theorem on continuous real
    functions~\cite[Theorem II.6.6]{Simpson:2009:Subsystems}.
    \item The Banach-Steinhaus theorem for a sequence of continuous linear operators
    on separable Banach spaces~\cite[theorem II.10.8]{Simpson:2009:Subsystems}.
    \item A weak version of G\"{o}del's completeness theorem (for a set of sentences, in a countable language,
    that is already closed under consequence).
    \item The existence of an algebraic closure for a countable field
    (but not its uniqueness)~\cite[Sections II.9.4--II.9.8]{Simpson:2009:Subsystems}
    \item The existence and uniqueness of the real closure
    of a countable ordered field~\cite[Sections II.9.5, II.9.7]{Simpson:2009:Subsystems}.
    \item The first-order part of \RCA{0} (the theorems of the system that do not involve any set variables)
    is the set of theorems of first-order Peano arithmetic with induction limited to $\Sigma^{0}_{1}$ formulas.
    It is provably consistent, as is \RCA{0}, in full first-order Peano arithmetic.
  \end{itemize}
  }
  \end{plQuote}

  \begin{plSection}{Robinson arithmetic~ \cite{wiki:RobinsonArithmetic}}

    Robinson Arithmetic referred to as $\RobinsonA$.

    (\TODO Is there an alternative not confusable with rationals $\Rationals$?)

    Domain is a set $\Set{N}$ (some version of the natural numbers).

    Equivalence: $\Eq{\RobinsonA}$

    Unary operation: $\mathsf{successor} : \Set{N} \rightarrow \Set{N}$

    Binary operations:
    $\Add{\RobinsonA} : \Set{N} \times \Set{N} \rightarrow \Set{N}$,
    $\Mul{\RobinsonA} : \Set{N} \times \Set{N} \rightarrow \Set{N}$\\

    Special element: $0_{\RobinsonA} \in \Set{N}$

    Axioms:
    \begin{enumerate}
      \item For all $n \in \Set{N}$,
      $\mathsf{successor}(n) \Neq{\RobinsonA} n$

      \item For all $n, m \in \Set{N}$,
      $\left( \mathsf{successor}(n) \Eq{\RobinsonA}
      \mathsf{successor}(m) \right)
      \implies \left( n \Eq{\RobinsonA} m \right)$

      \item If $n \Neq{\RobinsonA} 0_{\RobinsonA}$,
      there exists $m \in \Set{N}$ such that
      $\mathsf{successor}(m) \Eq{\RobinsonA} n$.

      \item For all $n \in \Set{N}$,
      $n \Add{\RobinsonA} 0_{\RobinsonA} \Eq{\RobinsonA} n$

      \item For all $n, m \in \Set{N}$,
      $\left( \mathsf{successor}(n) \Add{\RobinsonA} m \right)
      \Eq{\RobinsonA}
      \mathsf{successor}(n \Add{\RobinsonA} m)$

      \item For all $n \in \Set{N}$,
      $n \Mul{\RobinsonA} 0_{\RobinsonA} \Eq{\RobinsonA} 0_{\RobinsonA}$

      \item For all $n, m \in \Set{N}$,
      $\left( \mathsf{successor}(n) \Mul{\RobinsonA} m \right)
      \Eq{\RobinsonA}
      \left(
      \left( n \Mul{\RobinsonA} m \right)  \Add{\RobinsonA} m)
      \right)$

    \end{enumerate}


  \end{plSection}%{Robinson arithmetic}

  \begin{plSection}{Induction for $\Sigma^{0}_{1}$ formulas}
  \end{plSection}%{Induction for $\Sigma^{0}_{1}$ formulas}

  \begin{plSection}{Comprehension for $\Delta^{0}_{1}$ formulas}
  \end{plSection}%{Comprehension for $\Delta^{0}_{1}$ formulas}

\end{plSection}%{{RCA}_{0}}

\end{plSection}%{Foundations}

%-----------------------------------------------------------------
\begin{plSection}{Natural numbers}

I'm going to take the \emph{natural numbers},
$\Naturals =
 [ \Set{N}=\left\{ 0, 1, 2, {\cdots}\right\}, \Add{\Naturals}]$,
and the usual notion of addition on them $\Add{\Naturals}$,
as given.

(You can start from less than this if you like.
What's essential is a starting element, eg, $0$,
and a {\pseudocodeFont next}
or {\pseudocodeFont increment} operation. 
See, for example, 
\citeAuthorYearTitle[section 3.1]{Feferman:1989:NumberSystems}.
However, this kind of axiomatic minimalism 
isn't my primary interest.
What I want to do instead is to motivate the mathematical
structures I describe by posing at least somewhat realistic
problems that they can be used to solve.)

In the case of $\Naturals$, I find it a bit difficult 
come up with something convincing without
jumping ahead to the rationals.
So, if this seems a bit too trivial,
 bear with me for the moment.

The natural numbers answer two kinds of questions:
\begin{itemize}
\item How many?
(cardinality)
\item What's next?
(or Which is bigger?)
(ordinality)
\end{itemize}

Addition, $\Add{\Naturals}$, answers questions like:
If I have $m$ loaves of bread,
and my friend has $n$, how many do we have all together?

The ordering tells us: Who has more bread?

\TODO This isn't very satisfying.
Look into origins of math, Egypt, Mesopotamia, etc, for better
examples?

$\Naturals$ is a \emph{commutative monoid},
which is a monoid where the operation is symmetric:
\begin{description}
\item[Elements:] $\Set{N}$
\item[Operation:] $\Add{\Naturals}$
\item[Identity:] $\Ide{\Add{\Naturals}}() \rightarrow \Zero{\Naturals}$
\item[Associative:] 
$(n_0 \Add{\Naturals} n_1) \Add{\Naturals} n_2 \;=\;
n_0 \Add{\Naturals} (n_1 \Add{\Naturals} n_2)$, 
for all $n_0, n_1, n_2 \in \Set{N}$
\item[Commutative:] $n_0 \Add{\Naturals} n_1 \;=\;
n_1 \Add{\Naturals} n_0$,
for all $n_0, n_1 \in \Set{N}$
\end{description}

\TODO: missing $<$, $\leq$ as operator.
Should this be \emph{ordered commutative monoid},
or deal with ordering separately?

Multiplication, $\Mul{\Naturals}$, can be derived from addition in the
obvious way.
This gives us another commutative monoid:
$[\Set{N}, \Mul{\Naturals}, \Ide{\Mul{\Naturals}}]$.

Monoids are examples of what sometimes are referred to 
as group-like structures~\cite{wiki:GroupLike},
but might be better called ``one set, one (binary) operation''
structures.

We can also consider $\Naturals$ as a ring-like
(one set, two operation)
structure, a \emph{commutative semi-ring}~\cite{wiki:Semiring}, 
using both $\Add{\Naturals}$ and $\Mul{\Naturals}$,
which then obeys the commutative monoid laws for
$\Add{\Naturals}$ and $\Mul{\Naturals}$, and two more laws in addition:
\begin{description}
\item[Distributive:] 
$(n_0 \Add{\Naturals} n_1) \; \Mul{\Naturals}\; n_2 \;=\;
(n_0 \Mul{\Naturals} n_2) \; \Add{\Naturals} \; (n_1 \Mul{\Naturals} n_2)$, \\
for all $n_0, n_1, n_2 \in \Set{N}$.
\item[Zero:] $\Ide{\Add{\Naturals}} \Mul{\Naturals} n 
\; = \; \Ide{\Add{\Naturals}}
\; = \; 0$ for all $n \in \Set{N}$
\end{description}

This is not particularly important, but it serves as 
a very simple example
of how multiple mathematical structures can be defined
over a single set.
This does have significant implications for implementations.
It's tempting, and a common mistake, 
to represent mathematical structures by types or classes.
We might have, say, a Java class that provides an
unbounded implementation of the natural numbers,
with methods for $\Add{\Naturals}$, $\Mul{\Naturals}$, 
$\Ide{\Add{\Naturals}}$,
and $\Ide{\Mul{\Naturals}}$.
But that locks us into the semi-ring view,
disallowing the monoid views.

\TODO In this particular case, that doesn't seem so bad,
but we will see more convincing examples later.

%-----------------------------------------------------------------
\begin{plSection}{Modular arithmetic}

Another example of why it would be a mistake to implement
$\Naturals$ as, in Java terms,
a class implementing some {\javaFont CommutativeMonoid} interface,
is the fact that there are many other important structures
to be defined on subsets of $\Set{N}$, which require
different definitions for $\Add{\Naturals}$ and $\Mul{\Naturals}$.

The too simple motivating question here might be:
It is the $20$th hour of the day.
What hour will it be $5$ hours
from now?

To deal with questions like this, we need to modify 
$\Add{\Naturals}$,
give up ordering, and get, as a result, two new \emph{families} of 
commutative monoids (or a family of semi-rings, 
if we prefer to think of of it that way).

Consider the \emph{intervals} in $\Set{N}$:
For $n_{0}\,\leq\, n_{1}$, 
$\left[ n_{0}, n_{1} \right] \; \defeq \;
\left\{ n_{0}, n_{0} + 1, \cdots , n_{1} \right\}$.

(\NOTE To avoid off-by-one errors, 
I prefer to define all integer intervals as half-open.
But that is a bit difficult without subtraction.

\NOTE Modular arithmetic is more often defined for the special 
case of $n_0=0$.
My reasons for this slightly more general
version should become apparent soon.)

Then we can define a new addition operator 
$\Add{ \Naturals[n_0,n_1] }$:

\TODO breakable option for plChunks

\begin{plAlgorithm}
[breakable=false]
{Modular addition}
{modularAddition}
\begin{lstlisting}[language=pseudocode]
(defn $\Add{\Naturals[n_0,n_1]}$ [a b]
  (assert ($\Leq{\Naturals}$ 0 $n_0$ a $n_1$))
  (assert ($\Leq{\Naturals}$ 0 $n_0$ b $n_1$))
  (let [step (fn [sum count]
                    (if ($\Eq{\Naturals}$ count b)
                      sum
                      (if ($\Eq{\Naturals}$ sum $n_1$)
                        (step $n_0$ ($\Add{\Naturals}$ 1 count))
                        (step ($\Add{\Naturals}$ 1 sum) ($\Add{\Naturals}$ 1 count)))))]
    (step a 0)))
\end{lstlisting}
\end{plAlgorithm}
(\NOTE This is not intended to be a practical algorithm,
more a specification using only $\Add{\Naturals}$.)

\TODO Change $\Add{\Naturals}$ to something like
$\mathsf{\Naturals{.}+}$, 
$\mathsf{\Naturals{:}+}$, 
$\Naturals\mathsf{{:}+}$, 
$\Naturals\mathsf{{\colon}+}$, 
$\Naturals\mathsf{/+}$, 
$\mathsf{(+ \, \Naturals)}$, 
$\mathsf{+(\Naturals)}$, 
to look more like a package prefix or field value?

We can then define the commutative semi-group 
(a monoid without identity~\cite{wiki:Semigroup})
$
\Naturals[n_0,n_1] =
\left[ 
\left[ n_{0}, n_{1} \right], 
\Add{\Naturals[n_0,n_1]}
\right]
$.
(If $n_0=0$, then it's a commutative monoid.)

(A perhaps cleaner, or ``more pure'' version of this
would distinguish the elements of $\Naturals[n_0,n_1]$
from the elements of $\Naturals$,
and define $\Add{\Naturals[n_0,n_1]}$ via an isomorphism.
I'm using the ambiguity typical of mathematics to ignore that
distinction.)

\end{plSection}%{Modular arithmetic}
%-----------------------------------------------------------------
\begin{plSection}{Implementations}

Implementing $\Naturals$ is straightforward 
(as long as we ignore time and space constraints).
All we need is an implementation of unbounded bit sequences:

Any natural number $n$ can be written as
\begin{equation}
n \; = \; \sum_{k=0} \beta_{k}(n) * 2^{k}
\end{equation}
where $\beta_{k}(n)$ is $0$ or $1$,
and exponentiation over natural numbers is defined from
$\Mul{\Naturals}$ in the obvious way.
I'll note without proof that for any $n$ there is a 
$k_{\max}(n)$ such that $\beta_{k}(n) \equiv 0$ for all
$k \geq k_{\max(n)}$.
In other words, all such sequences are finite. 

Note also that we can define addition on bit sequences in the
obvious way, to match addition of the corresponding numbers.

The set of finite bit sequences 
$\Space{B} = \left\{[\beta_{0}, \beta_{1}, \cdots, 
\beta_{k_{\max}]} \right\}$
 is a 
\emph{representation} of $\Naturals$
in the sense that we can define an invertible function,
$\beta : \Naturals \rightarrow \Space{B}$,
(an \emph{isomorphism}) between $\Naturals$ and 
finite bit sequences
that preserves addition (and ordering and identity):
\begin{equation}
\beta \left( a \,\Add{\Naturals} \, b\right)
\;=\;
\beta ( a ) \, \Add{\Space{B}} \, \beta ( b )
\end{equation}
(again without proof).

A proof-of-concept implementation in Java,
supporting just the commutative monoid over addition, 
can be found at
\href{https://github.com/palisades-lakes/nzqr/blob/main/src/main/java/nzqr/java/numbers/UnboundedNatural.java}
{\javaFont UnboundedNatural.java}.
This implements $\Naturals$ with an unbounded sequence
of {\javaFont int} 32 bit words, treated as unsigned.
Adding two such sequences of $1073741809$ words
takes more than $20$ minutes and 
runs out of memory in a 56g JVM.

A more practical implementation,
that offers better performance for most purposes,
can be found at: 
\href{https://github.com/palisades-lakes/nzqr/blob/main/src/main/java/nzqr/java/numbers/BoundedNatural.java}
{\javaFont BoundedNatural.java}.
However, 
like \href{https://github.com/openjdk/jdk/blob/master/src/java.base/share/classes/java/math/BigInteger.java}
{\javaFont java.math.BigInteger},
its range is bounded, so it only supports a subset of $\Naturals$.

{\javaFont BoundedNatural} and {\javaFont BigInteger}
both use {\javaFont int[]} arrays to hold their bits.
The maximum size of an array in Java is limited
by the fact that the JVM 
requires array sizes and indexes to be specified 
with {\javaFont int} values.
(See \citeAuthorYearTitle[section 6.5]{LindholmEtAl:2021:JVMS16}.)
This limits us to at most {\javaFont Integer.MAX\_VALUE} elements.
However, the actual maximum array length is strictly less than that,
by an amount that appears to be implementation dependent.
In my particular setup, attempting to create an 
{\javaFont int[Integer.MAX\_VALUE-1]} results in:
``{\javaFont java.lang.OutOfMemoryError: 
Requested array size exceeds VM limit}'',
even though there is plenty of memory available to hold 
an array of that size.
(I haven't been able to find any clear discussion of this in
\citeAuthorYearTitle{GoslingEtAl:2021:JLS16}.)

There is an even smaller limit on the size of numbers that can be
represented by {\javaFont BoundedNatural} and 
{\javaFont java.math.BigInteger},
due to the fact that both allow access to arbitrary bits
indexed by {\javaFont int}.
This means the maximum length bit sequence is 
{\javaFont Integer.MAX\_VALUE},
and the maximum number of words is then
{\javaFont Integer.MAX\_VALUE>>5}.

(We could increase this a bit, since the bit indexes need to fit
in an {\javaFont int}, but the number of bits could be $1$ more
than the maximum index, or {\javaFont Integer.MAX\_VALUE + 1L}.
This would however, complicate the code, requiring a fair amount
of mixing of {\javaFont int} and {\javaFont long})

{\javaFont BoundedNatural}, like {\javaFont java.math.BigInteger},
deals with operations that would overflow the supported range
by throwing an exception, 
so it doesn't implement a monoid or semi-ring 
or anything like that.

\TODO Consider whether our foundational definition of a procedure
should include error exit as a possible outcome, eg,
$ f : \Set{X} \rightarrow \Set{Y}$ may
return $y\in \Set{Y}$, throw an exception,
or never halt.

An alternative would be to overflow into an unbounded 
representation, which would be more work that I want to do 
at present.
In practice, time and space constraints limit how much useful
computation can be done close to the overflow limit.

\TODO Use JMH to measure cost of arithmetic with
{\javaFont UnboundedNatural},
{\javaFont BoundedNatural},
and primitives.

\TODO Should this discussion be moved to the integer section?
So that signed numbers make more sense? On the other hand, 
it is a simpler to just stick to one operation:
the $\Add{\Naturals}$ monoid.


Most C-family languages offer primitive fixed length
unsigned integers, with $8$, $16$, $32$, $64$, and 
sometimes $128$ bits.
Arithmetic on these numbers is implemented with one or a few
instructions, XXXX times faster than possible with something
like {\javaFont UnboundedNatural} or
{\javaFont BoundedNatural}.
What is implemented is actually modular arithmetic,
so each of these primitive types corresponds to a valid
semi-ring.
aHowever, I think this is poorly understood by many developers.
I suspect many lurking bugs due to unchecked overflows.

Java, on the other hand, doesn't support unsigned numbers.
The reasoning behind that is a bit obscure,
and the brief explanations in various places on the web
are unconvincing:

\begin{plQuote}
{\citeAuthorYearTitle{RitchieStroustrupGosling:2000:CFamily}}{}%
{%
Q: Programmers often talk about the advantages and disadvantages 
of programming in a ''simple language.''  
What does that phrase mean to you, and is [C/C++/Java] 
a simple language in your view? 

\ldots

Gosling: For me as a language designer, 
which I don't really count myself as these days, 
what ''simple'' really ended up meaning was could I expect 
J. Random Developer to hold the spec in his head. 
That definition says that, for instance, Java isn't---and 
in fact a lot of these languages end up 
with a lot of corner cases, things that nobody really understands. 
\emph{Quiz any C developer about unsigned, 
and pretty soon you discover 
that almost no C developers actually understand what goes on 
with unsigned, what unsigned arithmetic is.}[emphasis added] 
Things like that made C complex. 
The language part of Java is, I think, pretty simple. 
The libraries you have to look up.}

\ldots
\end{plQuote}

I don't think this is correct.
My guess is that few developers are confused about
what unsigned values are.
Some may not get modular arithmetic,
and so don't understand what happens
operations exceed the supported range,
but the real confusion comes from mixed arithmetic, particularly
when mixing unsigned and signed values, as mentioned below:

\begin{plQuote}
{\citeAuthorYearTitle{JDK-4879804}}{}%
{%
A DESCRIPTION OF THE REQUEST: (Ranjith Mandala)

I've seen that there have been several requests 
for unsigned integral types in Java, 
and you have generally blown these off 
in a fairly cavalier manner. 
Usually you report that these types are unnecessary 
and can be worked around. In some cases you are correct, 
and I can write code that chews up CPU time 
masking and shifting bits around, 
using the char type as an unsigned temporary bucket, etc. 
However, I'm dealing with a stream of bytes 
that contains numeric values. 
I have to grab a few bytes at a time 
and make a number out of them---an integer, a long, whatever. 
Obviously, some of those bytes may have the high-order bit set, 
not because they are negative numbers 
but because that's how they fit in that particular byte 
in the sequence. 
Java's insistence on treating each byte as a signed value, 
and therefore doing sign-extension for me, is a major pain.

What this all boils down to is this: 
as a language vendor you should not be continually telling 
your user community that they are wrong 
when they request support from the language. 
Support the unsigned types because it is a reasonable thing to do, 
and stop pontificating. 
This is a major oversight in the language, and quite frankly, 
I'm amazed that you think it's okay to waste CPU run time 
rather than CPU compile time!

Joe Darcy added a comment - 2003-06-16 17:00

BT2:EVALUATION

If you want to more conveniently convert bytes to int or long, 
consider using nio byte buffers and viewing the byte buffer 
as an int or long buffer instead.

A parallel family of unsigned types was deliberately 
omitted from Java to avoid the confusion 
of combining signed and unsigned values in arithmetic expressions. 
That said, there are times
 when having unsigned arithmetic operations would be convenient. 
 However, if the bits of an int or long are interpreted 
 as unsigned two's complement numbers, 
 the output of add/subtract and multiply are 
 the same as if the numbers are interpreted as signed values 
 (this is a features of two's complement arithmetic). 
 Therefore, if you track signed-ness yourself, 
 you only need a separate unsigned divide method 
 and some conversion methods to have all the operations.}
\end{plQuote}

The real problem is trying to have one size fits all arithmetic.

When you have nested sets of numbers,you can get away with
automatic ``promotion'', 
although I think even there it is a mistake.
In other words, the set of numbers representable by
{\javaFont unsigned int} is a subset of those
representable by {\javaFont unsigned long},
so arithmetic mixing the two can be defined by identifying
with the {\javaFont unsigned int} values 
with the corresponding {\javaFont unsigned long}s,
and then using {\javaFont unsigned long} operations everywhere.

However, even in this simple case, mixed arithmetic is 
a source of lurking bugs.
There may be subexpressions involving only 
{\javaFont unsigned int} which would overflow/wraparound.
Might re-arrangements of the code may change whether the
wraparound happens or not, so a given intermediate value
may be $0$ in some cases, and $1+2^{32}$ in others.

I think the key problem is an invisible redefinition of 
``{\javaFont +}''.
Expressions in {\javaFont unsigned byte} implicitly define
{\javaFont +} as $\Add{\Naturals[0,0xFF]}$.
Change one argument to {\javaFont unsigned short} 
and {\javaFont +} is quietly redefined as
$\Add{\Naturals[0,0xFFFF]}$.
I suspect this leads to many developers forgetting about the
fact that the arithmetic is modular, and ignoring the possibility
of overflow.

The situation is worse, but not fundamentally different,
when you mixed number representations where neither set of
values is a subset of the other.
{\javaFont unsigned byte} is an implementation of
$\Naturals[0,255]$;
{\javaFont signed byte} implements $\Naturals[-128,127]$;
the union is $\Naturals[-128,255]$.
What should $+$ mean in that case?
I don't think anyone would find modular addition
on $[-128,255]$ an obvious choice (plus it would be expensive to 
implement).
The C standard, that Gosling was complaining about,
interprets the twos-complement bits of the signed number
as unsigned, effectively for signed/unsigned byte:
\begin{equation}
u + s \; \;=
\begin{cases}
\, u + s & \text{if } s \geq 0 \\
\, u + s + 256 & \text{if } s < 0 \\
\end{cases}
\end{equation}
which makes even less sense as ``generic addition''.

\TODO C-standard ref?

\NOTE Java's exclusion of unsigned integer types doesn't
eliminate the problem---the same issue arises in arithmetic mixing
{\javaFont long} and {\javaFont double},
where, despite it being described as a ``widening'' conversion~\cite[section 5.1.2]{GoslingEtAl:2021:JLS16},
the set of numbers representable in neither of the two types
is a subset of the other.

So then. what could/should $+$ mean---when given operands
representing different subsets of $\Naturals$?
\begin{itemize}
  \item No overflow: Implement {\javaFont +} as 
  $\Add{\Naturals}$. In other words, whenever the value of 
  {\javaFont a + b} would be outside the subset of $\Naturals$
  covered by their implementations, return an implementation
  that can handle that value. 
  
  This could be done statically,
  {\javaFont 
  ((unsigned int) a) + ((unsigned int) b) -> unsigned long},
  for all {\javaFont unsigned int} {\javaFont a} and {\javaFont b},
  or dynamically, where the returned type depends whether 
  {\javaFont a + b} fits in an {\javaFont unsigned int} or not.
  
  Either way is very expensive. The static version would quickly
  promote all but the simplest expressions to an impractical
  unbounded representation.
  The dynamic version requires giving up primitive arithmetic
  for object-based. (at least, I don't know of any language 
  that supports dynamically changing which primitive type is
  returned from a function, though I think I can see how 
  that might be done in some kind of tagged architecture.)
  
  \TODO JMH code to measure the cost of static/dynamic promotion,
  at least for a proof-of-concept implementation.
  
  \item Overflow exceptions: 
  \href{https://github.com/palisades-lakes/nzqr/blob/main/src/main/java/nzqr/java/numbers/BoundedNatural.java}
{\javaFont BoundedNatural} and 
\href{https://github.com/openjdk/jdk/blob/master/src/java.base/share/classes/java/math/BigInteger.java}
{\javaFont BigInteger}
both behave this way.
  
  Java offers this in some cases (for signed integers)
  with {\javaFont Math.addExact}.
  
  This still has significant cost relative to instruction level
  modular addition.
  
  \TODO JHM code to measure this, at least p-o-c.
  
  \item All arithmetic is modular, 
  using a specific $\Add{\Naturals[n_0,n_1]}$.
  Mixing arguments from implementations covering different subsets
  of $\Naturals$ requires explicit conversion to a monoid
  $\Naturals[n_0,n_1]$ that includes all of them.
  The danger here is that it's easy to forget that the arithmetic
  is modular, because the $+$ we learned in elementary school
  wasn't, and it's also easy to lose track of exactly which
  $\Naturals[n_0,n_1]$ we are operating in 
  (though requiring explicit conversion helps.)
  
  One way to think about this is that it is exposing a fundamental
  difference between mathematics (as a language) and programming
  languages. 
  Mathematics, as practiced is full of ambiguity---one 
  of its great strengths I would argue.
  We can view $3$ as simultaneously an element of $\Naturals$
  and any of the modular subsets of $\Naturals$ that contain it.
  Or we can think of $\Naturals[n_0,n_1]$ as isomorphic to 
  a subset of $\Naturals$, with a very distinct addition operator.
  
  When we start talking about implementation, on the other hand,
  we have to make a choice. 
  
  Are {\javaFont unsigned byte},
  {\javaFont unsigned short}, {\ldots} are to be viewed as
  elements of distinct monoids (or semi-rings),
  with natural isomorphisms between matching subsets?
  Then requiring explicit conversion makes sense.
  
  Or do we want to think of them all as elements of one big set?
  Then, to get one big monoid, we need to pick one definition
  of {\javaFont +}, and the modular $\Add{\Naturals[n_0,n_1]}$
  for the enclosing $\Naturals[n_0,n_1]$ 
  is the only choice that works.
  This is equivalent to automatic promotion to the largest 
  word size, as discussed above.
  
\end{itemize}

\end{plSection}%{Implementations}
%-----------------------------------------------------------------
\end{plSection}%{Natural numbers}
%-----------------------------------------------------------------
\begin{plSection}{Integers}

A common situation that leads to new mathematical structures
is this:
We have a problem that can be expressed in an existing structure,
but doesn't have a solution that can be expressed 
in that structure, at least, not in all cases.

This is how we get from the natural numbers to the positive
and negative integers.
Posing an example problem requiring
negative integers will no doubt seem silly,
since nearly all developers already know the answer.
But perhaps it will make the core process clearer,
without the potential distraction that later,
more complex problems may add.

And it's worth noting that, even though just about everyone learns
about negative numbers in elementary school,
and they were in use by at least the $200$s (in Han China),
european mathematicians resisted them
an invalid, or cheating, up to the $1800$s. 
(See \citeAuthorTitle{wiki:NegativeNumber}.)

Suppose I have $12$ loaves and I've promised 
my friend $8$. How many can I give to someone else?
Abstracting that a bit,
we have
$a = b + c$
where $a,b,c \in \Naturals$ and I know 
$a$, the number of loaves I have, 
and $b$, the number I've promised,
but not $c$, the number that will remain.
I am also assuming I have more loaves than I have promised.
Pretending that we have never heard of subtraction,
we might solve for $c$ with a simple search:

\begin{plAlgorithm}
{Reinventing subtraction}
{reinventSubtraction}
\begin{lstlisting}[language=pseudocode]
(defn $\Sub{\Naturals}$ [^$\Naturals$ a ^$\Naturals$ b]
  (assert ($\Leq{\Naturals}$ b a))
  (let [step (fn [c]
                    (if ($\Eq{\Naturals}$ a ($\Add{\Naturals}$ b c))
                      c
                      (step ($\Add{\Naturals}$ 1 c))]
    (step 0)))
\end{lstlisting}
\end{plAlgorithm}

Given an implementation for $\Naturals$, or some subset that
includes $a,b,c$, and a little thought, 
we can come up with a more efficient, direct algorithm 
for $\Sub{\Naturals}$, one that doesn't involve trying all 
possibilities.

What we end up with is a new operator (binary function),
which, unlike the previous ones, is only defined on a 
subset:
$\Sub{\Naturals}:
\left\{ \, [a,b] \in \Naturals{\times}\Naturals 
\; \middle|  \; a \geq b \, \right\} \, \rightarrow \, \Naturals$.

%-----------------------------------------------------------------
\begin{plSection}{Implementations}
%-----------------------------------------------------------------
\begin{plSection}{Primitives}

\end{plSection}%{Primtives}
%-----------------------------------------------------------------
\begin{plSection}{\texorpdfstring{{\javaFont java.math.BigInteger}}{java.math.BigInteger}}

\end{plSection}%{\javaFont BigInteger}
%-----------------------------------------------------------------
\begin{plSection}{\texorpdfstring{{\javaFont clojure.lang.BigInt}}{clojure.lang.BigInt}}

{\javaFont clojure.lang.BigInt} is a minimal wrapper around
{\javaFont java.math.BigInteger}, primarily to provide
a large integer implementation whose {\javaFont hashCode}
matches the {\javaFont hashCode} for {\javaFont Long}s
representing the same integer value---{\javaFont BigInteger}s
does not~\cite[page 438]{EmerickCarperGrand:2012:ClojureProgramming}.

Because {\javaFont BigInt} doesn't provide an API comparable to
{\javaFont BigInteger}, it's much less useful than
{\javaFont BigInteger}, and probably best reserved as a wrapper
for keys in hashtables or sets the might be either
{\javaFont Long} or {\javaFont BigInteger}---though even in that
case, it would be simpler to just coerce all the keys to
{\javaFont BigInteger}.

(\TODO 
More about hashcodes and implications of equivalence across 
implementations/representations.

The fundamental problem is Java's assumption
that ``correct'' {\javaFont equals}
and {\javaFont hashCode} can be defined for a single class
in isolation.
For any (mathematical) set of objects, there are many possible
notions of equivalence (\TODO what's this called?).
The ``correct'' one depends on the problem being solved.
(\TODO examples).
A ``correct'' hashcode preserves the equivalence:
\begin{equation}
a \approx b \; \implies \; \text{hashcode}(a) = \text{hashcode}(b)
\end{equation}
(What's the right way to say this?)

Check {\javaFont Byte}s, {\javaFont Short}s, or
{\javaFont Integer}s representing the same integer value.
Also, the value of getting {\javaFont BigInt}s
{\javaFont hashCode} ``right'' is diminished by uncertainty
about exactly what implementation will be returned when
doing arithmetic in Clojure.) 

{\javaFont BigInt} also ``optimizes'' the case where the value
would fit in $64$ two's complement bits (ie a {\javaFont long}).
It's not clear to me that this actually improves performance.

\end{plSection}%{\ clojure.lang.BigInt}
%-----------------------------------------------------------------
\end{plSection}%{Implementations}
%-----------------------------------------------------------------
\end{plSection}%{Integers}
%-----------------------------------------------------------------
\begin{plSection}{Rationals}

%-----------------------------------------------------------------
\begin{plSection}{B-adic numbers}


\end{plSection}%{B-adic numbers}
%-----------------------------------------------------------------
\begin{plSection}{Floating point}


\end{plSection}%{Floating point}
%-----------------------------------------------------------------
\begin{plSection}{Implementations}
%-----------------------------------------------------------------
\begin{plSection}{\texorpdfstring{\javaFont RationalFloat}{RationalFloat}}

\end{plSection}%{\javaFont RationalFloat}
%-----------------------------------------------------------------
\begin{plSection}{Primitives}

\end{plSection}%{Primtives}
%-----------------------------------------------------------------
\end{plSection}%{Implementations}
%-----------------------------------------------------------------
\end{plSection}%{Rationals}
%-----------------------------------------------------------------
\begin{plSection}{Classical Reals}


\end{plSection}%{Classical Reals}
%-----------------------------------------------------------------
\begin{plSection}{Computable Reals}

\cite{Aberth:2001:ComputableCalculus,
BratkaHertlingWeihrauch:2008:TutorialComputableAnalysis,
Weihrauch:2000:ComputableAnalysis}

%-----------------------------------------------------------------
\begin{plSection}{Implementations}
\end{plSection}%{Implementations}
%-----------------------------------------------------------------
\end{plSection}%{Computable Reals}
%-----------------------------------------------------------------
%\BeginAppendices
%\input{typesetting}
%-----------------------------------------------------------------
\end{document}
%-----------------------------------------------------------------
